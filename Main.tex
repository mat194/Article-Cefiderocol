% Options for packages loaded elsewhere
\PassOptionsToPackage{unicode}{hyperref}
\PassOptionsToPackage{hyphens}{url}
\PassOptionsToPackage{dvipsnames,svgnames,x11names}{xcolor}
%
\documentclass[
  number,
  review,
  1p]{elsarticle}

\usepackage{amsmath,amssymb}
\usepackage{iftex}
\ifPDFTeX
  \usepackage[T1]{fontenc}
  \usepackage[utf8]{inputenc}
  \usepackage{textcomp} % provide euro and other symbols
\else % if luatex or xetex
  \usepackage{unicode-math}
  \defaultfontfeatures{Scale=MatchLowercase}
  \defaultfontfeatures[\rmfamily]{Ligatures=TeX,Scale=1}
\fi
\usepackage{lmodern}
\ifPDFTeX\else  
    % xetex/luatex font selection
\fi
% Use upquote if available, for straight quotes in verbatim environments
\IfFileExists{upquote.sty}{\usepackage{upquote}}{}
\IfFileExists{microtype.sty}{% use microtype if available
  \usepackage[]{microtype}
  \UseMicrotypeSet[protrusion]{basicmath} % disable protrusion for tt fonts
}{}
\makeatletter
\@ifundefined{KOMAClassName}{% if non-KOMA class
  \IfFileExists{parskip.sty}{%
    \usepackage{parskip}
  }{% else
    \setlength{\parindent}{0pt}
    \setlength{\parskip}{6pt plus 2pt minus 1pt}}
}{% if KOMA class
  \KOMAoptions{parskip=half}}
\makeatother
\usepackage{xcolor}
\setlength{\emergencystretch}{3em} % prevent overfull lines
\setcounter{secnumdepth}{5}
% Make \paragraph and \subparagraph free-standing
\makeatletter
\ifx\paragraph\undefined\else
  \let\oldparagraph\paragraph
  \renewcommand{\paragraph}{
    \@ifstar
      \xxxParagraphStar
      \xxxParagraphNoStar
  }
  \newcommand{\xxxParagraphStar}[1]{\oldparagraph*{#1}\mbox{}}
  \newcommand{\xxxParagraphNoStar}[1]{\oldparagraph{#1}\mbox{}}
\fi
\ifx\subparagraph\undefined\else
  \let\oldsubparagraph\subparagraph
  \renewcommand{\subparagraph}{
    \@ifstar
      \xxxSubParagraphStar
      \xxxSubParagraphNoStar
  }
  \newcommand{\xxxSubParagraphStar}[1]{\oldsubparagraph*{#1}\mbox{}}
  \newcommand{\xxxSubParagraphNoStar}[1]{\oldsubparagraph{#1}\mbox{}}
\fi
\makeatother


\providecommand{\tightlist}{%
  \setlength{\itemsep}{0pt}\setlength{\parskip}{0pt}}\usepackage{longtable,booktabs,array}
\usepackage{calc} % for calculating minipage widths
% Correct order of tables after \paragraph or \subparagraph
\usepackage{etoolbox}
\makeatletter
\patchcmd\longtable{\par}{\if@noskipsec\mbox{}\fi\par}{}{}
\makeatother
% Allow footnotes in longtable head/foot
\IfFileExists{footnotehyper.sty}{\usepackage{footnotehyper}}{\usepackage{footnote}}
\makesavenoteenv{longtable}
\usepackage{graphicx}
\makeatletter
\newsavebox\pandoc@box
\newcommand*\pandocbounded[1]{% scales image to fit in text height/width
  \sbox\pandoc@box{#1}%
  \Gscale@div\@tempa{\textheight}{\dimexpr\ht\pandoc@box+\dp\pandoc@box\relax}%
  \Gscale@div\@tempb{\linewidth}{\wd\pandoc@box}%
  \ifdim\@tempb\p@<\@tempa\p@\let\@tempa\@tempb\fi% select the smaller of both
  \ifdim\@tempa\p@<\p@\scalebox{\@tempa}{\usebox\pandoc@box}%
  \else\usebox{\pandoc@box}%
  \fi%
}
% Set default figure placement to htbp
\def\fps@figure{htbp}
\makeatother

\makeatletter
\@ifpackageloaded{caption}{}{\usepackage{caption}}
\AtBeginDocument{%
\ifdefined\contentsname
  \renewcommand*\contentsname{Table of contents}
\else
  \newcommand\contentsname{Table of contents}
\fi
\ifdefined\listfigurename
  \renewcommand*\listfigurename{List of Figures}
\else
  \newcommand\listfigurename{List of Figures}
\fi
\ifdefined\listtablename
  \renewcommand*\listtablename{List of Tables}
\else
  \newcommand\listtablename{List of Tables}
\fi
\ifdefined\figurename
  \renewcommand*\figurename{Figure}
\else
  \newcommand\figurename{Figure}
\fi
\ifdefined\tablename
  \renewcommand*\tablename{Table}
\else
  \newcommand\tablename{Table}
\fi
}
\@ifpackageloaded{float}{}{\usepackage{float}}
\floatstyle{ruled}
\@ifundefined{c@chapter}{\newfloat{codelisting}{h}{lop}}{\newfloat{codelisting}{h}{lop}[chapter]}
\floatname{codelisting}{Listing}
\newcommand*\listoflistings{\listof{codelisting}{List of Listings}}
\makeatother
\makeatletter
\makeatother
\makeatletter
\@ifpackageloaded{caption}{}{\usepackage{caption}}
\@ifpackageloaded{subcaption}{}{\usepackage{subcaption}}
\makeatother

\usepackage[]{natbib}
\bibliographystyle{elsarticle-num}
\usepackage{bookmark}

\IfFileExists{xurl.sty}{\usepackage{xurl}}{} % add URL line breaks if available
\urlstyle{same} % disable monospaced font for URLs
\hypersetup{
  pdftitle={Exploring the evidence for use of cefiderocol as part of combination therapies: a systematic review of in vitro, in vivo, and clinical studies},
  pdfauthor={Marco Meroi; Juan Antonio del Castillo Polo; Rebecca Scardellato; Alessandra Nazeri; Alessia Savoldi; Renata Da Costa; Laura Piddock; Jennifer Cohn; Evelina Tacconelli; Matteo Morra; Elda Righi},
  pdfkeywords={cefiderocol, combination therapy, systematic review, in
vitro studies, in vivo studies, human studies},
  colorlinks=true,
  linkcolor={blue},
  filecolor={Maroon},
  citecolor={Blue},
  urlcolor={Blue},
  pdfcreator={LaTeX via pandoc}}


\setlength{\parindent}{6pt}
\begin{document}

\begin{frontmatter}
\title{Exploring the evidence for use of cefiderocol as part of
combination therapies: a systematic review of \emph{in vitro}, \emph{in
vivo}, and clinical studies}
\author[1]{Marco Meroi%
%
}

\author[2]{Juan Antonio del Castillo Polo%
%
}

\author[1]{Rebecca Scardellato%
%
}

\author[1]{Alessandra Nazeri%
%
}

\author[1]{Alessia Savoldi%
%
}

\author[3]{Renata Da Costa%
%
}

\author[3]{Laura Piddock%
%
}

\author[3]{Jennifer Cohn%
%
}

\author[1]{Evelina Tacconelli%
%
}

\author[1]{Matteo Morra%
\corref{cor1}%
}
 \ead{matteo.morra@univr.it} 
\author[1]{Elda Righi%
%
}


\affiliation[1]{organization={Division of Infectious Diseases,
Department of Diagnostics and Public Health, University of Verona,
Verona, Italy},,postcodesep={}}
\affiliation[2]{organization={Servicio de Microbiología, Hospital
Universitario Ramón y Cajal and Instituto Ramón y Cajal de Investigación
Sanitaria (IRYCIS), Madrid, Spain},,postcodesep={}}
\affiliation[3]{organization={Global Antibiotic Research and Development
Partnership (GARDP), Geneva, Switzerland},,postcodesep={}}

\cortext[cor1]{Corresponding author}











        





\begin{keyword}
    cefiderocol \sep combination therapy \sep systematic review \sep in
vitro studies \sep in vivo studies \sep 
    human studies
\end{keyword}
\end{frontmatter}
    

\section{\texorpdfstring{\textbf{Introduction}}{Introduction}}\label{introduction}

Cefiderocol is a novel siderophore--cephalosporin that has emerged as a
promising agent against multidrug-resistant Gram-negative bacteria
(MDR-GNB), including carbapenem-resistant \emph{Enterobacterales},
\emph{Pseudomonas aeruginosa}, and \emph{Acinetobacter} species
\citep{ito2016, hackel2018}.

While most recent guidelines for usage of cefiderocol consistently
recommend monotherapy~ for CRE and CRPA infections, when susceptibility
is confirmed in vitro, there is no consensus on indications for CRAB
infections \citep{paul2022, tamma2024}. The 2024 Infectious Diseases
Society of America (IDSA) guidance suggests the use of cefiderocol as an
alternative agent for CRAB infections, specifically for cases refractory
to other treatments or when patients cannot tolerate other options,
always within a combination regimen, on the basis that there is
insufficient evidence that supports the effectiveness of any single
molecule when used alone. Conversely, the 2022~ European Society of
Clinical Microbiology and Infectious Diseases (ESCMID) guidelines
conditionally recommend against the use of cefiderocol for CRAB
infections, mainly on the basis of results of a single study showing
increased mortality in patients receiving cefiderocol compared to those
receiving the best available treatment \citep{paul2022, tamma2024}.

The 2019 COHERENCE survey, involving 1012 experts worldwide, identified
the intent to improve efficacy (81\% of respondents) and to prevent
resistance (51\% of respondents) as the main reasons for the use of
combination therapy in the treatment of CR-GNB infections, although many
experts acknowledged the poor quality of supporting data and the lack of
standardization of regimens \citep{carrara2022}. Similar findings
emerged from the CLEARER 2023 global survey, focusing on cefiderocol
perception and prescribing attitudes, in which combination therapy was
frequently preferred for CRAB and MBL-producing GNB infections,
particularly to minimise resistance development and in the case of
critical illness (personal data, not shown).

This systematic review aims to assess the efficacy of cefiderocol-based
combination therapy compared with monotherapy in treating
carbapenem-resistant Gram-negative bacterial infections. Evidence is
synthesized across \emph{in vitro} studies assessing synergy, \emph{in
vivo} models evaluating bacterial clearance, and clinical trials
examining mortality, clinical cure, and microbiological eradication.

\section{Methods}\label{methods}

\subsection{\texorpdfstring{\textbf{Search strategy and eligibility
criteria}}{Search strategy and eligibility criteria}}\label{search-strategy-and-eligibility-criteria}

A Medline-based search was conducted from 1st of January 2015 until 31st
of January 2025 using ``cefiderocol{[}tw{]}'' or ``S-649266{[}tw{]}'' as
keywords to identify \emph{in vivo, in vitro}, and clinical studies on
humans reporting data on cefiderocol as part of combination regimens.
Only studies published in the English language were included. The
screening of papers was conducted independently by three authors
(Ma.Mo., J.DC, and R.S.) and disagreements were addressed by involving a
third reviewer (E.R. or Ma.Me.). If eligibility could not be determined,
the full article was retrieved. References of reviews and original
publications were hand searched for further eligible studies.For in
vitro studies, an additional web-based search of abstracts presented at
the annual European Congress of Clinical Microbiology and Infectious
Diseases (ECCMID) and IDWeek was conducted by systematically screening
the ESCMID eLibrary and the Open Forum Infectious Diseases library,
respectively, over the same period. All \emph{in vivo} and \emph{in
vitro} studies were included, except for case reports, irrespective of~
the technique performed to evaluate synergies.

Regarding human studies, randomized trials, observational comparative
studies, and non-comparative cohort studies were included, provided that
they reported on cefiderocol-based combination therapy. Case reports and
case series with fewer than 10 patients were excluded.

\subsection{\texorpdfstring{\textbf{Data extraction and
synthesis}}{Data extraction and synthesis}}\label{data-extraction-and-synthesis}

A standardised data-extraction method was used to record relevant
features from each study into a database, including year of publication,
country, study design, number of isolates/animals/patients included,
type of bacterium cultured, type of antibiotic molecules tested,
antibiotic susceptibility profile, type of infection, and setting. The
susceptibility profile of isolates to the antibiotic(s) tested,
including cefiderocol, were extracted, when available, according to the
microbiological guidelines for breakpoint interpretation adopted by the
individual study.~ Results were presented narratively, grouped by
pathogen species for \emph{in vitro} and \emph{in vivo} studies and by
study design for clinical studies.

Limited to the clinical studies, a pair-wise meta-analysis was conducted
comparing clinical and microbiological outcomes in cefiderocol
monotherapy vs.~cefiderocol-based combination therapy. Only studies with
more than 25 patients were considered for the pooled analysis; case
series and non-comparative reports were excluded. Full methodology is
detailed in the \textbf{Supplementary Material}.

\subsection{\texorpdfstring{\textbf{Quality
assessment}}{Quality assessment}}\label{quality-assessment}

Two independent reviewers (Ma.Mo. and Ma.Me.) assessed the study quality
using two different scores according to the study design\textbf{:}
Cochrane Risk of Bias tool for randomized trials and the
Newcastle-Ottawa Scale for observational cohort studies.

\section{Results}\label{results}

A total of 1496 citations were retrieved. From these, 191 articles were
included for further screening and 66 (34 \emph{in vitro,} 2 \emph{in
vivo}, and 30 clinical studies) met the eligibility criteria and were
included in the evidence assessment. Four ECCMID abstracts and three
IDWeek abstracts were identified.~ Three \emph{in vivo} models were
described by the authors as part of \emph{in vitro} articles.

PRISMA flow chart for data search and extraction, screening, and
selection process is displayed in \textbf{Figure 1}. Results are
reported by type of studies.

\subsection{\texorpdfstring{\textbf{\emph{In vitro}}
\textbf{studies}}{In vitro studies}}\label{in-vitro-studies}

A total of 34 studies were included, 11 of them with ≥30 isolates and 20
with cefiderocol-resistant isolates. Eighteen of them included
combinations with avibactam and 7 with sulbactam, whereas xeruborbactam
and meropenem were the molecules evaluated with more isolates, 325 and
274, respectively. Four methods were used: time-kill assays (12
studies), checkerboard analysis (13), broth microdilution (9), and
gradient diffusion strip crossing (7), with 9 studies employing ≥1
method.

Overall, combinations with β-lactamase inhibitors, particularly novel
agents, emerged as the most promising ones in most cases (\textbf{Table
1}). Some experiments included different bacterial species. In a study
with 82 GNB, meropenem showed synergy for all species, while amikacin
and colistin were not synergistic in \emph{P. aeruginosa}
\citep{tsuji2016}. In another evaluation, no synergy was demonstrated
with piperacillin-tazobactam, meropenem-vaborbactam, or
imipenem-relebactam and synergy with fosfomycin and
ceftazidime-avibactam was found only in two isolates
\citep{palombo2023}, but ceftazidime-avibactam also caused a ≥8-fold MIC
decrease in 32 of 33 non-NDM-producing GNB \citep{kohira2020}. Synergy
was found for ceftazidime-avibactam, ceftolozane-tazobactam, and
meropenem for non-NDM-producing isolates from SIDERO-WT-2014 study
\citep{yamano2020}. Cefiderocol-sulbactam enhanced cefiderocol activity
in 6 GNB \citep{lewis2024} and positive interaction but no synergy was
found for glycine in 10 GNB \citep{giordano2024}.~ Zidebactam, another
BLI, was found synergistic in \emph{P. aeruginosa} and in
Enterobacterales, and avibactam, zidebactam, taniborbactam, and
relebactam combinations increased cefiderocol susceptibility rates more
than 80\% in \emph{A. baumannii} complex \citep{leterrier2024, xu2025}.
Xeruborbactam significantly improved cefiderocol activity in two studies
with 160 and 165 139 CR GNB \citep{hara2025, hara2025a}.

\subsubsection{\texorpdfstring{\textbf{\emph{Acinetobacter baumannii}}
\textbf{complex}}{Acinetobacter baumannii complex}}\label{acinetobacter-baumannii-complex}

Synergy with amikacin was shown in 7 amikacin-resistant isolates~
\citep{tsuji2016, abdul-mutakabbir2020} and with tigecycline,
minocycline, meropenem, and sulbactam in cefiderocol-resistant isolates
\citep{abdul-mutakabbir2020}. Besides this, tigecycline showed synergism
in 123 isolates, confirmed \emph{in vivo} with a \emph{Galleria
mellonella} model \citep{ni2022}, as well as other tetracycline
analogues (minocycline, tigecycline, eravacycline, and omadacycline) in
CRAB isolates, verifying the eravacycline synergism in a neutropenic
murine thigh-infection model \citep{yin2025}. No synergy was observed
with colistin, tigecycline, or fosfomycin in 15 CRAB\emph{,} potentially
due to the use of the gradient diffusion strip crossing technique
\citep{stolfa2021}.

Among β-lactams, cefiderocol synergism with meropenem was demonstrated
in CRAB but was not reported in other three studies
\citep{tsuji2016, yamano2020, abdul-mutakabbir2020, ni2022}. Avibactam
and sulbactam were found synergistic in 2 cefiderocol-resistant
PER-producing isolates, but not tazobactam, avibactam, vaborbactam, or
relebactam in 7 OXA-carbapenemase-producing isolates
\citep{yamano2020, bianco2022}. Cefiderocol-zidebactam was shown as a
promising combination by an \emph{in silico} analysis
\citep{gopikrishnan2023} and durlobactam lowered the
cefiderocol-MIC\textsubscript{50} and MIC\textsubscript{90} of 66
isolates but sulbactam did not \citep{huband2023}. In another study,
only the cefiderocol-sulbactam-tigecycline combination reached synergy
in an extensively drug-resistant (XDR) \emph{A. baumannii} strain from a
patient with VAP \citep{kobic2022}. Of note, exposure of \emph{A.
baumannii} to cefiderocol and sulbactam or avibactam led to the
selection of resistant strains \citep{wong2024}. Cefiderocol, polymyxin
B, and rifampicin showed synergistic effects by artificial intelligence
but ampicillin-sulbactam exhibited significant antagonistic interaction
\citep{you2025}\textbf{.}

\subsubsection{\texorpdfstring{\textbf{Enterobacterales}}{Enterobacterales}}\label{enterobacterales}

Synergies with BLIs have been the most frequently evaluated,
demonstrated in tazobactam, avibactam, vaborbactam, and relebactam in
three OXA-48 producers but very low synergy with tazobactam, relebactam,
and avibactam and no synergy with vaborbactam was found in 7 MBL
producers \citep{bianco2022}. In addition, no synergy with avibactam was
shown in 10 double-carbapenemase producers \citep{bianco2022a}, but it
was found synergistic in 20 NDM-producing \emph{E. cloacae} isolates
\citep{göpel2024}.

In \emph{K. pneumoniae}, synergy with avibactam was demonstrated in
different experiments including KPC and NDM-producers
\citep{bianco2022, bianco2022a, boattini2023, moon2023} and it produced
the most remarkable fold reduction in the MIC compared to other BLIs in
34 isolates \citep{daoud2023}. However, cefiderocol activity was not
improved by the addition of avibactam alone but only when aztreonam or
dipicolinic acid was also added in 1 isolate producing NDM-1, OXA-232
and CTX-M-15 and with a mutation in catecholate-siderophore receptor .
Relebactam and vaborbactam were found synergic in 18 KPC-producers
\citep{bianco2022a}, and synergy with meropenem was demonstrated in one
KPC-producer, but was not confirmed for amikacin \citep{tsuji2016}.

\subsubsection{\texorpdfstring{\textbf{\emph{Pseudomonas
aeruginosa}}}{Pseudomonas aeruginosa}}\label{pseudomonas-aeruginosa}

No synergy was shown for avibactam, vaborbactam, or tazobactam in 1
MBL-producer, but it was found for relebactam in this isolate
\citep{bianco2022} and two VIM-producers, and for imipenem-relebactam in
two IMP-producers \citep{boattini2023, granata2025}. Cotreatment with
imipenem resulted in synergistic bactericidal activity in 5 isolates
\citep{ferretti2024} but no synergy for meropenem or amikacin was found
in one VIM-producer \citep{tsuji2016}. However, colistin significantly
improved cefiderocol efficacy against biofilms in 2 isolates in an
\emph{in vitro} pharmacodynamic model \citep{elhaj2024}. Vancomycin
addition to cefiderocol was indifferent in 2 isolates
\citep{schilling2024} and colloidal bismuth citrate showed synergism,
confirmed in a murine acute pneumonia model, and increased cefiderocol
efficacy against biofilm formation, restored susceptibility in a
cefiderocol-resistant isolate and significantly increased survival rate
and decreased the bacterial load \emph{in vivo} \citep{wang2023}.

\subsubsection{\texorpdfstring{\textbf{\emph{Stenotrophomonas
maltophilia}}}{Stenotrophomonas maltophilia}}\label{stenotrophomonas-maltophilia}

Different synergy values were found with levofloxacin, minocycline,
polymyxin B, and trimethoprim-sulfamethoxazole among 9 isolates
\citep{biagi2020}.~

\subsection{\texorpdfstring{\textbf{\emph{In vivo}}
\textbf{studies}}{In vivo studies}}\label{in-vivo-studies}

Three \emph{in vitro} studies also included an \emph{in vivo} model
\citep{ni2022, yin2025, wang2022}. Ni \emph{et al.} demonstrated
increased survival in a \emph{Galleria mellonella} model for both
cefiderocol-resistant CRAB (treated with cefiderocol and either colistin
or tigecycline) and for cefiderocol-susceptible CRAB (treated with
cefiderocol and tigecycline). In this study all the isolates were
susceptible to colistin \citep{ni2022}. Wang \emph{et al.} demonstrated
increased survival for combination of cefiderocol with CBS in a \emph{P.
aeruginosa} murine pneumonia model and Yin \emph{et al}. verified the
cefiderocol-eravacycline bacterial cell reduction in a neutropenic
murine thigh-infection model \citep{yin2025, wang2023}. Two studies were
retrieved with an exclusive focus on an in vivo model
\citep{ding2024, gill2023}. Ding \emph{et al}. have evaluated
nanomedicine elements in a mouse intra-abdominal infection model.
Deferiprone-loaded layered double hydroxide-based therapy associated
with cefiderocol suppressed the emergence of drug-resistant bacteria and
enhanced the bactericidal efficacy \citep{ding2024}.

Cefiderocol efficacy and resistance were tested in combination with
ceftazidime-avibactam, ampicillin-sulbactam, or meropenem against
\emph{A. baumannii} complex isolates using human-simulated regimens in a
murine thigh infection model. The isolates were classified according to
cefiderocol MICs (3 had MIC of 2 mg/L, 2 of 8 mg/L, and 10 of ≥ 32 mg/L)
and some included VEB and PER β-lactamases. When cefiderocol was
combined with ceftazidime-avibactam or ampicillin-sulbactam in isolates
that were cefiderocol- and ceftazidime-avibactam-resistant, mean
log\textsubscript{10} CFU/thigh reductions of −3.75 ± 0.37 and −3.55 ±
0.50 were observed, respectively. The combination with meropenem was
less effective. Post-treatment, combination with ceftazidime-avibactam
and ampicillin-sulbactam did not show MICs increase; conversely,
cefiderocol monotherapy was associated with elevated MICs in all 3
isolates with baseline MICs of 2 mg/L. \emph{In vitro} disk stacking
assessments demonstrated a return of all tested isolates treated with
ceftazidime-avibactam or ampicillin-sulbactam combinations to the
Clinical \& Laboratory Standards Institute (CLSI) intermediate
breakpoint \citep{gill2023}.

\subsection{\texorpdfstring{\textbf{Clinical
studies}}{Clinical studies}}\label{clinical-studies}

Three prospective,28 retrospective observational studies, and one
randomised trial were included for the analysis. The risk of bias~ was
either moderate or high for cohort studies and (\textbf{Supplementary
Table 3}) and moderate for the randomised trial. Cohort studies were
often monocentric
\citep{dalfino2023, falcone2022, mazzitelli2023, russo2024, bavaro2023, piccica2023, gavaghan2023, frattari2024, oliva2024},
focusing exclusively
\citep{dalfino2023, falcone2022, mazzitelli2023, russo2024, bavaro2023}
or mainly \citep{piccica2023, gavaghan2023, elghali2024, clancy2024} on
CRAB infections, and some used cefiderocol as a rescue therapy
\citep{mazzitelli2023, bavaro2023, giannella2023}. Only four articles,
including 142, 41, 38, 200 and 45 patients respectively, compared
cefiderocol monotherapy with combination therapy as a primary outcome
\citep{piccica2023, palermo2023, buonomo2024, giacobbe2024}. Over 15
antibiotic combinations were used in different studies, including the
association with antibiotics without \emph{in vitro} activity against
target CR-GNB. Detailed data and results~ of the single studies are
summarized in \textbf{Table 2.}

In the exploratory meta-analysis comparing monotherapy with cefiderocol
and combination therapy, the pooled 30-day all cause mortality was
computed on 854 patients (9 studies), while pooled clinical and
microbiological cure was computed on 799 patients (nine studies) and 454
patients (five studies), respectively. Pooled mortality was
significantly higher among patients receiving cefiderocol combination
therapy (OR = 1.42, 95\% CI: 1.03--1.95). No significant differences
were observed between combination therapy and monotherapy in terms of
clinical cure (OR = 0.79, 95\% CI: 0.48--1.31) or microbiological
eradication (OR = 0.96, 95\% CI: 0.57--1.63).

\section{\texorpdfstring{\textbf{Discussion}}{Discussion}}\label{discussion}

This study comprehensively assessed the state of art of the
cefiderocol-based combination treatments against CR-GNB infections
encompassing \emph{in vitro} and \emph{in vivo studies} as well as
current clinical practice. \textbf{Figure 2} summarizes the main
findings and limitations of the studies included.

\emph{In vitro} experiments with cefiderocol combinations may elucidate
which combinations could be synergistic for both cefiderocol-susceptible
and resistant isolates. Results from these studies can be equally
useful, the former to outline the mechanisms that reduce cefiderocol
MICs, and the latter to investigate how to overcome cefiderocol
resistance. However, there are some limitations hampering the
comparability and therefore the applicability of results into practice.
First, different techniques were employed for assessing the synergy.
Reduction of CFU/mL in time was assessed by time-kill assays, whereas
MIC changes were evaluated using the fractional inhibitory concentration
index in most experiments. MICs were obtained by broth microdilution in
checkerboard analysis. Moreover, conclusions from studies using MIC
gradient test strips should be interpreted carefully and might not be as
valid as other techniques \citep{bonnin2022}. Second, several \emph{in
vitro} experiments have analysed different types of resistant bacteria
in an aggregate manner, making it difficult to interpret the results as
distinct resistance patterns were found. Third, the number of isolates
varied widely across studies, with the majority including small
experiments. Even when considering these drawbacks, the most relevant
synergies identified for clinical application were BLIs, tigecycline,
meropenem, ceftazidime-avibactam and, to a lesser extent, colistin, and
fosfomycin. An increasing number of \emph{in vitro} studies, involving
considerable numbers of isolates, suggests that cefiderocol combined
with BLIs may be active against selected pathogens. Of the BLIs
evaluated, sulbactam has shown promising activity when combined with
cefiderocol against CRAB. Although sulbactam--durlobactam with imipenem
is currently recommended as first-line therapy by the IDSA guidance for
this pathogen, the optimal combination partner, particularly for MBL
producing strains, remains undefined. Based on current \emph{in vitro}
data, cefiderocol may therefore represent a good candidate to combine
with sulbbactam-dorlubactam. Another promising option is cefiderocol in
combination with xeruborbactam, a bicyclic boronate β-lactamase
inhibitor currently tested in combination with cefiderocol in a phase 1
clinical trial. The two recent in vitro studies by Hara et al, which
together tested 325 cefiderocol-resistant isolates (165
\emph{Enterobacterales} and 160 \emph{Acinetobacter baumannii}),
represent the largest datasets to date evaluating cefiderocol
combinations. Both studies employed the reference broth microdilution
method in iron-depleted cation-adjusted Mueller-Hinton broth, following
CLSI guidelines. To date, no other combination has been tested on such a
large number of cefiderocol-resistant isolates using this standardised
approach \citep{hara2025a, hara2025b}.

Evidence from \emph{in vivo} studies remains scarce and provides limited
insights. Although some studies have assessed \emph{in vivo} the
synergistic effects observed \emph{in vitro} using various methods,
these investigations have typically involved only a small number of
isolates, limiting the strength of the conclusions that can be drawn.

The number of observational studies on the use of cefiderocol has
increased in recent years, reflecting the expanding clinical use of this
agent and the growing interest in optimizing its clinical application.
Cefiderocol used in combination, in particular with fosfomycin as
companion agent, appears promising
\citep{falcone2022, russo2024, russo2023}, especially for the treatment
of CRAB infections. Our meta-analysis suggests that cefiderocol
combination therapy does not confer a consistent advantage over
monotherapy, irrespective of type of pathogen considered. While pooled
estimates slightly favored combination therapy for microbiological
eradication, confidence intervals were wide and subgroup differences
were generally not statistically significant or based on limited study
numbers.

Nonetheless, a major limitation in evaluating the efficacy of
cefiderocol-based combination therapy in human studies lies in the
heterogeneity and methodological quality of the available data. In many
cohorts, cefiderocol was combined with various companion antibiotics
leading to outcome aggregation across non-uniform regimens. This may
have masked potential benefits by including combinations with
antagonistic or toxic effects. The lack of detailed microbiological
characterization, particularly regarding resistance mechanisms like
MBLs, further limits the ability to interpret treatment effects across
different bacterial profiles. Moreover, most observational studies do
not adequately adjust for confounding variables. Patients receiving
combination therapy are often more severely ill or have a higher risk of
mortality at baseline, which may bias outcomes in favor of monotherapy.
As a result, the apparent lack of benefit observed in some studies may
reflect confounding by indication rather than a true absence of effect.

\section{\texorpdfstring{\textbf{Conclusions}}{Conclusions}}\label{conclusions}

Currently available evidence is of low quality, highly heterogeneous,
and insufficient to support firm conclusions about the clinical
applicability of cefiderocol-based combination therapies. Nevertheless,
these findings offer valuable insight into potential directions for
future research. Systematic \emph{in vitro} evaluation of
cefiderocol-based combinations against well-characterized CR-GNB, using
validated and standardized synergy testing methods, is warranted.
Particular emphasis should be placed on combinations with BLIs, which
are increasingly supported by emerging preclinical data. At the same
time, continued investigation of agents such as tigecycline,
carbapenems, fosfomycin, colistin and selected non-antibiotic adjuvants
remains essential to fully explore their synergistic potential.

Well-structured clinical trials and observational studies are needed to
define the role of cefiderocol-based combinations, with systematic
assessment of clinical outcomes and microbiological endpoints, including
resistance emergence. Given the promising in vitro activity and the
paucity of clinical trials or high-quality observational studies
assessing cefiderocol--BLI combinations, in vivo and clinical evaluation
is needed to better define their therapeutic role.

\subsubsection{\texorpdfstring{\textbf{Funding}}{Funding}}\label{funding}

This work was part of the CLEARER project (Current cLinical scEnARios
for the use of cefidErocol in multidrug-Resistant infections: a case
study-based approach) supported by the Global Antibiotic Research and
Development Partnership (GARDP).

\subsubsection{\texorpdfstring{\textbf{Transparency
declarations}}{Transparency declarations}}\label{transparency-declarations}

No conflicts of interest to declare

\subsubsection{References}\label{references}

\renewcommand{\bibsection}{}
\bibliography{bibliography.bib}





\end{document}
