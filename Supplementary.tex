% Options for packages loaded elsewhere
\PassOptionsToPackage{unicode}{hyperref}
\PassOptionsToPackage{hyphens}{url}
\PassOptionsToPackage{dvipsnames,svgnames,x11names}{xcolor}
%
\documentclass[
  number,
  review,
  1p]{elsarticle}

\usepackage{amsmath,amssymb}
\usepackage{iftex}
\ifPDFTeX
  \usepackage[T1]{fontenc}
  \usepackage[utf8]{inputenc}
  \usepackage{textcomp} % provide euro and other symbols
\else % if luatex or xetex
  \usepackage{unicode-math}
  \defaultfontfeatures{Scale=MatchLowercase}
  \defaultfontfeatures[\rmfamily]{Ligatures=TeX,Scale=1}
\fi
\usepackage{lmodern}
\ifPDFTeX\else  
    % xetex/luatex font selection
\fi
% Use upquote if available, for straight quotes in verbatim environments
\IfFileExists{upquote.sty}{\usepackage{upquote}}{}
\IfFileExists{microtype.sty}{% use microtype if available
  \usepackage[]{microtype}
  \UseMicrotypeSet[protrusion]{basicmath} % disable protrusion for tt fonts
}{}
\makeatletter
\@ifundefined{KOMAClassName}{% if non-KOMA class
  \IfFileExists{parskip.sty}{%
    \usepackage{parskip}
  }{% else
    \setlength{\parindent}{0pt}
    \setlength{\parskip}{6pt plus 2pt minus 1pt}}
}{% if KOMA class
  \KOMAoptions{parskip=half}}
\makeatother
\usepackage{xcolor}
\setlength{\emergencystretch}{3em} % prevent overfull lines
\setcounter{secnumdepth}{5}
% Make \paragraph and \subparagraph free-standing
\makeatletter
\ifx\paragraph\undefined\else
  \let\oldparagraph\paragraph
  \renewcommand{\paragraph}{
    \@ifstar
      \xxxParagraphStar
      \xxxParagraphNoStar
  }
  \newcommand{\xxxParagraphStar}[1]{\oldparagraph*{#1}\mbox{}}
  \newcommand{\xxxParagraphNoStar}[1]{\oldparagraph{#1}\mbox{}}
\fi
\ifx\subparagraph\undefined\else
  \let\oldsubparagraph\subparagraph
  \renewcommand{\subparagraph}{
    \@ifstar
      \xxxSubParagraphStar
      \xxxSubParagraphNoStar
  }
  \newcommand{\xxxSubParagraphStar}[1]{\oldsubparagraph*{#1}\mbox{}}
  \newcommand{\xxxSubParagraphNoStar}[1]{\oldsubparagraph{#1}\mbox{}}
\fi
\makeatother


\providecommand{\tightlist}{%
  \setlength{\itemsep}{0pt}\setlength{\parskip}{0pt}}\usepackage{longtable,booktabs,array}
\usepackage{calc} % for calculating minipage widths
% Correct order of tables after \paragraph or \subparagraph
\usepackage{etoolbox}
\makeatletter
\patchcmd\longtable{\par}{\if@noskipsec\mbox{}\fi\par}{}{}
\makeatother
% Allow footnotes in longtable head/foot
\IfFileExists{footnotehyper.sty}{\usepackage{footnotehyper}}{\usepackage{footnote}}
\makesavenoteenv{longtable}
\usepackage{graphicx}
\makeatletter
\newsavebox\pandoc@box
\newcommand*\pandocbounded[1]{% scales image to fit in text height/width
  \sbox\pandoc@box{#1}%
  \Gscale@div\@tempa{\textheight}{\dimexpr\ht\pandoc@box+\dp\pandoc@box\relax}%
  \Gscale@div\@tempb{\linewidth}{\wd\pandoc@box}%
  \ifdim\@tempb\p@<\@tempa\p@\let\@tempa\@tempb\fi% select the smaller of both
  \ifdim\@tempa\p@<\p@\scalebox{\@tempa}{\usebox\pandoc@box}%
  \else\usebox{\pandoc@box}%
  \fi%
}
% Set default figure placement to htbp
\def\fps@figure{htbp}
\makeatother

\makeatletter
\@ifpackageloaded{caption}{}{\usepackage{caption}}
\AtBeginDocument{%
\ifdefined\contentsname
  \renewcommand*\contentsname{Table of contents}
\else
  \newcommand\contentsname{Table of contents}
\fi
\ifdefined\listfigurename
  \renewcommand*\listfigurename{List of Figures}
\else
  \newcommand\listfigurename{List of Figures}
\fi
\ifdefined\listtablename
  \renewcommand*\listtablename{List of Tables}
\else
  \newcommand\listtablename{List of Tables}
\fi
\ifdefined\figurename
  \renewcommand*\figurename{Figure}
\else
  \newcommand\figurename{Figure}
\fi
\ifdefined\tablename
  \renewcommand*\tablename{Table}
\else
  \newcommand\tablename{Table}
\fi
}
\@ifpackageloaded{float}{}{\usepackage{float}}
\floatstyle{ruled}
\@ifundefined{c@chapter}{\newfloat{codelisting}{h}{lop}}{\newfloat{codelisting}{h}{lop}[chapter]}
\floatname{codelisting}{Listing}
\newcommand*\listoflistings{\listof{codelisting}{List of Listings}}
\makeatother
\makeatletter
\makeatother
\makeatletter
\@ifpackageloaded{caption}{}{\usepackage{caption}}
\@ifpackageloaded{subcaption}{}{\usepackage{subcaption}}
\makeatother

\usepackage[]{natbib}
\bibliographystyle{elsarticle-num}
\usepackage{bookmark}

\IfFileExists{xurl.sty}{\usepackage{xurl}}{} % add URL line breaks if available
\urlstyle{same} % disable monospaced font for URLs
\hypersetup{
  pdftitle={Exploring the evidence for use of cefiderocol as part of combination therapies: a systematic review of in vitro, in vivo, and clinical studies},
  pdfauthor={Marco Meroi; Juan Antonio del Castillo Polo; Rebecca Scardellato; Alessandra Nazeri; Alessia Savoldi; Renata Da Costa; Laura Piddock; Jennifer Cohn; Evelina Tacconelli; Matteo Morra; Elda Righi},
  colorlinks=true,
  linkcolor={blue},
  filecolor={Maroon},
  citecolor={Blue},
  urlcolor={Blue},
  pdfcreator={LaTeX via pandoc}}


\setlength{\parindent}{6pt}
\begin{document}

\begin{frontmatter}
\title{Exploring the evidence for use of cefiderocol as part of
combination therapies: a systematic review of \emph{in vitro}, \emph{in
vivo}, and clinical studies \\\large{Supplementary material} }
\author[1]{Marco Meroi%
%
}

\author[2]{Juan Antonio del Castillo Polo%
%
}

\author[1]{Rebecca Scardellato%
%
}

\author[1]{Alessandra Nazeri%
%
}

\author[1]{Alessia Savoldi%
%
}

\author[3]{Renata Da Costa%
%
}

\author[3]{Laura Piddock%
%
}

\author[3]{Jennifer Cohn%
%
}

\author[1]{Evelina Tacconelli%
%
}

\author[1]{Matteo Morra%
\corref{cor1}%
}
 \ead{matteo.morra@univr.it} 
\author[1]{Elda Righi%
%
}


\affiliation[1]{organization={Division of Infectious Diseases,
Department of Diagnostics and Public Health, University of Verona,
Verona, Italy},,postcodesep={}}
\affiliation[2]{organization={Servicio de Microbiología, Hospital
Universitario Ramón y Cajal and Instituto Ramón y Cajal de Investigación
Sanitaria (IRYCIS), Madrid, Spain},,postcodesep={}}
\affiliation[3]{organization={Global Antibiotic Research and Development
Partnership (GARDP), Geneva, Switzerland},,postcodesep={}}

\cortext[cor1]{Corresponding author}











        





\end{frontmatter}
    

\section{Exploratory Meta-analysis}\label{exploratory-meta-analysis}

\subsection{Materials and methods}\label{materials-and-methods}

Randomized controlled trials (RCTs) and observational cohort studies
enrolling at least 25 patients were included. Case series, case reports,
and studies with fewer than 25 total participants were excluded from the
analysis. Details regarding the search strategy, screening and data
extraction process are provided in the main manuscript.

\subsection{Quality of evidence}\label{quality-of-evidence}

Quality assessment was performed using the Cochrane risk-of-bias tool
for RCTs and the Newcastle-Ottawa Scale for observational cohort
studies.

\begin{longtable}[]{@{}ll@{}}
\caption{\textbf{Table S1}: Robins-2 assessment of risk of bias--
Clinical trial, Bassetti et al 2020 \citep{bassetti2021}. Note that risk
of Bias is assessed focusing on the comparison between cefiderocol
monotherapy and combination therapy and not the primary endpoint of the
study.}\tabularnewline
\toprule\noalign{}
Criteria & \textbf{Evaluation} \\
\midrule\noalign{}
\endfirsthead
\toprule\noalign{}
Criteria & \textbf{Evaluation} \\
\midrule\noalign{}
\endhead
\bottomrule\noalign{}
\endlastfoot
Randomisation process & - High risk \\
Deviations from Intended interventions Signaling & - High risk \\
Missing outcomes & - Low risk \\
Measurement of outcome & - Low risk \\
Selection of reported results & - Some concernes \\
\textbf{Overall} & - Some concernes \\
\end{longtable}

\begin{longtable}[]{@{}
  >{\raggedright\arraybackslash}p{(\linewidth - 4\tabcolsep) * \real{0.1506}}
  >{\raggedright\arraybackslash}p{(\linewidth - 4\tabcolsep) * \real{0.4826}}
  >{\raggedright\arraybackslash}p{(\linewidth - 4\tabcolsep) * \real{0.3629}}@{}}
\caption{Risk of bias: low 6-7 stars, moderate 4-5 stars, high 1-3
stars. Note that risk of Bias is assessed focusing on the comparison
between cefiderocol monotherapy and combination therapy that, in most of
the cases, is not the primary outcome of interest of the
studies.}\tabularnewline
\toprule\noalign{}
\begin{minipage}[b]{\linewidth}\raggedright
\textbf{Criteria}
\end{minipage} & \begin{minipage}[b]{\linewidth}\raggedright
\textbf{Acceptable}

\textbf{(star awarded)}
\end{minipage} & \begin{minipage}[b]{\linewidth}\raggedright
\textbf{Unacceptable}

\textbf{(star not awarded)}
\end{minipage} \\
\midrule\noalign{}
\endfirsthead
\toprule\noalign{}
\begin{minipage}[b]{\linewidth}\raggedright
\textbf{Criteria}
\end{minipage} & \begin{minipage}[b]{\linewidth}\raggedright
\textbf{Acceptable}

\textbf{(star awarded)}
\end{minipage} & \begin{minipage}[b]{\linewidth}\raggedright
\textbf{Unacceptable}

\textbf{(star not awarded)}
\end{minipage} \\
\midrule\noalign{}
\endhead
\bottomrule\noalign{}
\endlastfoot
Representativeness of exposed cohort & Truly representative OR somewhat
representative & No description \\
Selection of non-exposed cohort & Drawn from the same community as the
exposed cohort & Drawn from a different source OR no description of the
derivation of the non-exposed cohort \\
Ascertainment of exposure & Secure records or directly measured &
Self-report OR unclear \\
Comparability & \begin{minipage}[t]{\linewidth}\raggedright
\begin{itemize}
\tightlist
\item
  Adjusted for age, sex, comorbidities.
\item
  Adjusted for known colonization or source control, or previous
  antibiotic therapy or time from diagnosis to antibiotic
\end{itemize}
\end{minipage} & No adjustment \\
Outcome of interest & Secure records or directly measured & Self-report
OR unclear \\
Adequacy of follow-up & Adjusted for missing data or follow-up
\textgreater{} 14 days. & No statement regarding missing data. No
follow-up after end of therapy \\
\end{longtable}

\subsection{Analysis}\label{analysis}

A meta-analysis was conducted using random-effects models with the
restricted maximum-likelihood (REML) estimator to account for
between-study variability. The analysed outcomes were 30-day all-cause
mortality, clinical and microbiological cure in patients treated with
cefiderocol combination therapy versus monotherapy. Effect sizes were
calculated as odds ratios (ORs) with 95\% confidence intervals (CIs).
When available, adjusted effect sizes were pooled using the inverse
variance method. In the absence of adjusted estimates, crude odds ratios
were calculated and included in the analysis.

A 95\% prediction interval was also displayed to estimate the range in
which the true effect of a new study is expected to lie. Heterogeneity
across studies was assessed using the Chi-squared test (with a p-value
\textless{} 0.1 indicating substantial heterogeneity) and the I²
statistic (with values \textgreater{} 50\% considered indicative of
moderate to high heterogeneity).

A subgroup analysis was performed based on the type of infection,
distinguishing between CRAB infections only and mixed MDR infections. To
assess whether treatment effects differed significantly between
subgroups, a Q-test for subgroup differences was performed. With two
subgroups, this test is based on one degree of freedom. A p-value
\textless{} 0.05 was considered indicative of a significant difference
in effect estimates.

Potential publication bias was evaluated through contour-enhanced funnel
plots, incorporating significance contours at \emph{p} \textless{} 0.1,
\emph{p} \textless{} 0.05, and \emph{p} \textless{} 0.01, as shown in
the plot legends.Egger's test for funnel plot asymmetry was not
performed, as none of the pooled analyses included 10 or more studies.
For the same reason, the trim and fill method was not applied.

Influence analysis was conducted to evaluate the robustness of the
results. Baujat plots were used to identify studies with the greatest
contribution to heterogeneity and effect size, and leave-one-out
analyses were performed to assess the influence of individual studies on
the overall pooled estimate (results not shown, available on github).

All analyses were conducted using R software (version 4.4.3). The meta
and dmetar packages were primarily used for meta-analysis calculations
and sensitivity analysis.

\subsection{\texorpdfstring{\textbf{Data
availability}}{Data availability}}\label{data-availability}

Data and analysis are available
at:\url{https://github.com/mat194/Cefiderocol-Meta-analysis}

\subsection{Results}\label{results}

\subsubsection{\texorpdfstring{\textbf{Does cefiderocol combination
therapy improve 30 days mortality compared to
monotherapy?}}{Does cefiderocol combination therapy improve 30 days mortality compared to monotherapy?}}\label{does-cefiderocol-combination-therapy-improve-30-days-mortality-compared-to-monotherapy}

\begin{center}
\pandocbounded{\includegraphics[keepaspectratio]{images/clipboard-384120512.png}}
\end{center}

A total of nine retrospective studies published between 2022 and 2024
were included in the meta-analysis, comprising both monocentric and
multicentric designs from Italy (n = 6), the USA (n = 1), France (n =
1), and an international collaboration (n = 1). Only one study
\citep{piccica2023} provided an adjusted effect estimate using a
propensity score inverse probability weighting method (aOR 1.11, 95\% CI
0.63--1.96). The remaining studies relied on univariate analyses. CRAB
was the most frequently targeted pathogen, although several studies also
included broader MDR organisms.

The test for subgroup differences was not statistically significant,
indicating no strong evidence of differential treatment effect between
the subgroups.

The influence analysis indicated that no single study disproportionately
affected the overall pooled estimate or heterogeneity. The Baujat plot
showed that onely \emph{Falcone, 2022} \citep{falcone2022} contributed
most to heterogeneity, while \emph{Piccica, 202}3 \citep{piccica2023}
had the greatest influence on the pooled result. However, leave-one-out
sensitivity analysis confirmed the stability of the overall effect
estimate, with no major change in heterogeneity (I² remained 0\% in all
iterations).

\begin{center}
\pandocbounded{\includegraphics[keepaspectratio]{images/clipboard-4124015486.png}}
\end{center}

The contour-enhanced funnel plot does not reveal clear evidence of
publication bias. The distribution of studies appears relatively
symmetrical around the central effect estimate, and most points fall
within non-significant regions, suggesting that any observed asymmetry
is unlikely to be due to selective reporting of statistically
significant results.

\subsubsection{\texorpdfstring{\textbf{Does cefiderocol combination
therapy improve clinical cure compared to
monotherapy?}}{Does cefiderocol combination therapy improve clinical cure compared to monotherapy?}}\label{does-cefiderocol-combination-therapy-improve-clinical-cure-compared-to-monotherapy}

\pandocbounded{\includegraphics[keepaspectratio]{images/clipboard-4070756505.png}}

None of the included studies reported adjusted effect estimates; all
results were derived from univariate analyses. The analysis includes one
randomized clinical trial, two prospective cohort studies, and six
retrospective observational studies.

The influence analysis revealed that no single study had a
disproportionate impact on the pooled effect estimate, although a few
studies contributed more to heterogeneity or had a slightly higher
influence. According to the Baujat plot, \emph{Dalfino, 2023}
\citep{dalfino2023} showed the highest contribution to heterogeneity,
while \emph{Giannella, 2023} \citep{giannella2023} had the greatest
influence on the pooled result.

The test for subgroup differences was not statistically significant,
suggesting no clear evidence of a differential treatment effect between
the two infection categories.

Leave-one-out analysis confirmed the robustness of the meta-analytic
findings. Omitting individual studies led to minimal shifts in the
pooled odds ratio, and none of the exclusions resulted in a
statistically significant change. Heterogeneity remained low to moderate
(I² range: 0\% to 43\%) throughout, indicating consistent findings
across studies.

\begin{center}
\pandocbounded{\includegraphics[keepaspectratio]{images/clipboard-3145531308.png}}
\end{center}

\subsubsection{\texorpdfstring{\textbf{Does cefiderocol combination
therapy improve microbiological cure compared to
monotherapy?}}{Does cefiderocol combination therapy improve microbiological cure compared to monotherapy?}}\label{does-cefiderocol-combination-therapy-improve-microbiological-cure-compared-to-monotherapy}

\pandocbounded{\includegraphics[keepaspectratio]{images/clipboard-135960318.png}}

None of the included studies reported adjusted effect estimates; all
outcomes were based on univariate comparisons. The analysis includes one
randomized clinical trial, and four retrospective observational studies.

The influence analysis showed that the overall pooled estimate was
robust, with no single study exerting a disproportionate influence on
the meta-analytic result. The Baujat plot identified \emph{Mazzitelli,
2023} as the most influential study on the pooled effect size, followed
by \emph{El Ghali, 2024} \citep{elghali2024} and \emph{Palermo, 2023}
\citep{palermo2023}. However, the contribution of all studies to overall
heterogeneity remained minimal.

The test for subgroup differences yielded a statistically significant
result, suggesting a potential difference in treatment effect between
the CRAB and mixed infection groups. However, this result must be
interpreted with caution. As noted in the Cochrane Handbook for
Systematic Reviews of Interventions, statistical tests for subgroup
differences can yield spurious significance when the number of studies
is small, and power to detect true differences is generally low. With
only two and three studies in the respective subgroups, the observed
significance may reflect random variation rather than a true
differential effect.

Leave-one-out sensitivity analysis confirmed the stability of the
results. Omitting any single study did not lead to significant shifts in
the pooled odds ratio. Heterogeneity remained low (I² = 0\%) throughout
all iterations, reinforcing the consistency of findings.

\begin{center}
\pandocbounded{\includegraphics[keepaspectratio]{images/clipboard-3107019296.png}}
\end{center}

\subsubsection{References}\label{references}

\renewcommand{\bibsection}{}
\bibliography{bibliography.bib}





\end{document}
